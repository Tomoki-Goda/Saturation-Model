\documentclass[12pt]{article}
\usepackage[a4paper,left=1in,right=1in]{geometry}
%\usepackage{merriweather}
\usepackage{hyperref}
\usepackage{amsfonts}
\usepackage{amsmath}
\usepackage{amssymb}
\title{Saturation Model Fitter\\ (ver.~3)\\User's Guide}
\author{Tomoki Goda}

%\newenvironment{entry}[1]{\textbf{\large #1:}\vspace{1pt}\hrule\begin{center}}{\end{center}\vspace{2mm}}
\newenvironment{entry}[1]{\textbf{\large #1:  }}{\\\vspace{3mm}\\}

\begin{document}
	\maketitle
%\section{Intro.}

%\section{Functions}
\section{gluons.hh}
\subsection{\texttt{class} Collinear\_Gluon }\label{CollinearGluon}

\begin{entry}{Description}
computes $xg(x,Q^2)$.\\
$xg(x,Q_0^2)=A_g x^\lambda_g(1-x)^{5.6}$
\end{entry} 
\begin{entry}{Constructor}
	Collinear\_Gluon()
\end{entry} 
\begin{entry}{Main function}
	\textbf{operator}()($x$,$Q^2$,$A_g$,$\lambda_g$)
\end{entry} 
\begin{entry}{Dependence}
polygamma.h, clenshaw.hh
\end{entry}	


\subsection{\texttt{class} Chebyshev\_Collinear\_Gluon}\label{ChebyshevCollinearGluon}
\begin{entry}{Description}
	Approximate $xg(x,Q^2)$ with 2D Chebyshev polynomials. 
\end{entry}
\begin{entry}{Constructor}
	Chebyshev\_Collinear\_Gluon()\\
	Global variable or Macro \texttt{N\_CHEB} is required to specify order of polynomials.
\end{entry}
\begin{entry}{Initialization}
	init($x_{\mathrm{min}}$, $x_{\mathrm{max}}$, $Q^2_{\mathrm{min}}$, $Q^2_{\mathrm{max}}$,$A_g$, $\lambda_g$ )\\
	Computes coefficients.\\
	set\_x($x$)\\
	Reduce 2D polynomial to 1D at specified $x$. 
\end{entry}
\begin{entry}{Main function}
	\textbf{operator}()($x$,$Q^2$)
	compute $xg(x,Q^2)$ from 2D polynomial or 1D if set\_x($x$) was run previously.(it can be reset by running set\_x($x$) with negative value. ) 
\end{entry}
\begin{entry}{Dependence}
	chebyseh.hh, Collinear\_Gluon~(\ref{CollinearGluon})
\end{entry}

\subsection{\texttt{class} Chebyshev1D\_Collinear\_Gluon}\label{Chebyshev1DCollinearGluon}
\begin{entry}{Description}
	Approximate $xg(x,Q^2)$ with 1D Chebyshev polynomials at fixed $x$. 
\end{entry}
\begin{entry}{Constructor}
	Chebyshev1D\_Collinear\_Gluon()\\
	Global variable or Macro \texttt{N\_CHEB} is required to specify order of polynomials.
\end{entry}
\begin{entry}{Initialization}
	init($x_{\mathrm{min}}$, $x_{\mathrm{max}}$, $Q^2_{\mathrm{min}}$, $Q^2_{\mathrm{max}}$,$A_g$, $\lambda_g$ )\\
	Computes coefficients.\\
	set\_x($x$)\\
	Record the value of $x$ and compute coefficients.
\end{entry}
\begin{entry}{Main function}
	\textbf{operator}()($x$,$Q^2$)\\
	compute $xg(x,Q^2)$ from 1D polynomial. If passed $x$ differs from the value at which approximation is done, it will print error.
\end{entry}
\begin{entry}{Dependence}
	chebyseh.hh, Collinear\_Gluon~(\ref{CollinearGluon})
\end{entry}
\subsubsection{\texttt{class} Interpolate\_Collinear\_Gluon}
No longer supported.

\section{r-formula.hh}
\subsection{Sigma}\label{Sigma}
\begin{entry}{Description}
	$\sigma_{\mathrm{dipole}}(x,r)$.
\end{entry}
\begin{entry}{Constructor}
	controlled by macro \texttt{MODEL}\\
	if MODEL == 1 (BGK)\\
		Sigma$<$cg$>$()\\
		where cg = is appropriate collinear gluon\\
		Collinear\_Gluon~(\ref{CollinearGluon}), Chebyshev\_Collinear\_Gluon~(\ref{ChebyshevCollinearGluon}),  Chebyshev1D\_Collinear\_Gluon~(\ref{Chebyshev1DCollinearGluon})
	else \\
		Sigma()\\
\end{entry}
\begin{entry}{Initialization}
	init(\textbf{const double *const }parameters)\\
	parameter array "parameters". see also Parameters.hh~(\ref{Parameters.hh}).\\

	set\_x(x)\\
	run cg.set\_x(x) for approx.ed collinear gluons (\texttt{SIGMA\_APPROX}$<$0).
	note, for \texttt{FREEZE\_QS2}$!=$0, set\_x($x$) will modify $x$ accordingly at this point and pass it to collinear gluon.
\end{entry}
\begin{entry}{Main function}
	\textbf{operator}()(x,r)\\
	computes $\sigma_{\mathrm{dipole}}(x,r)$.
\end{entry}
\section{Interpolation-dipole.hh}\label{Interpolationdipole}

\subsection{Laplacian\_Sigma}\label{LaplacianSigma}
\begin{entry}{Description}
	Approximates $\sigma_{\mathrm{dipole}}(x,r)$ at given $x$ with cubic spline. 
	return value is controlled by macros and most importantly returns $rJ_0(rk)\nabla^2_r \sigma$ or $rkJ_1(rk)\frac{\partial \sigma}{\partial r}$.
	Essentially a class to compute integrand for the Hankel transform;
	\begin{equation}
		\int^{r_{\mathrm{max}}}_{r_{\mathrm{min}}} dr r J_0(rk)\nabla^2_{r}\sigma_{\mathrm{dipole}}(x,r)
	\end{equation}
\end{entry}
\begin{entry}{Constructor}
	Laplacian\_Sigma$<$sigma$>$(sigma)
	where sigma is the dipole cross section (\ref{Sigma})
\end{entry}
\begin{entry}{Initialization}
	init(int N,const double * parameters, char mode)\\
	\texttt{N} is the number of points to be taken for the 1D grid. (for cubic spline, it should be around 500?)
	\texttt{parameters} is 1d array containing parameters for $\sigma$. \texttt{mode} is `l' or `s'.  (`l' stands for laplacian, and `s' for sigma (for when laplacian can be computed analytically )). \\
	set\_x($x$) will set $x$ in the components ($\sigma$, $xg$ etc.) and compute 1D grid of dipole cross section.
\end{entry}
\begin{entry}{Main function}
	\textbf{operator}()(\textbf{double} $\rho$, \textbf{std::vector$<$double$>$} arg)\\
	$\rho$ is $r$ or $\frac{r}{1+r}$, use macro \texttt{R\_CHANGE\_VAR}.
	arg = \{$k_t^2$,$x$\} 
\end{entry}
\begin{entry}{macro \texttt{IBP}}
	Use \texttt{IBP} = 0,1,2 to control which of the following integrand you want; 
	\begin{align}
		\int^{r_{\mathrm{max}}}_{r_{\mathrm{min}}} dr r J_0(rk)\nabla^2_{r}\sigma_{\mathrm{dipole}}(x,r)\\
		\left[ r J_0(rk)\frac{\partial\sigma_{\mathrm{dipole}}(x,r)}{\partial r}\right]^{r_{\mathrm{max}}}_{r_{\mathrm{min}}}
		+k\int^{r_{\mathrm{max}}}_{r_{\mathrm{min}}} dr r J_1(rk)\frac{\partial\sigma_{\mathrm{dipole}}(x,r)}{\partial r}\\
			\left[ r J_0(rk)\frac{\partial\sigma_{\mathrm{dipole}}(x,r)}{\partial r}+
			rk J_1(rk)\sigma_{\mathrm{dipole}}(x,r)
			 \right]^{r_{\mathrm{max}}}_{r_{\mathrm{min}}}
			-k^2\int^{r_{\mathrm{max}}}_{r_{\mathrm{min}}} dr r J_0(rk)\sigma_{\mathrm{dipole}}(x,r)
	\end{align}
	use constant(double $r$, std::vector$<$double$>$ arg ) for the first term, arg is the same but $r$ is $r_{\mathrm{max}}$ or $r_{\mathrm{min}}$ (it is $r$ not $\rho$!!). 
	also \texttt{NS}=1 will insert $\exp(-r^2/500)$ in the both first and second term, to suppress large $r$. 
	for \texttt{NS}=2 exists (not well maintained), which is  equivalent to inserting that exponent in the integral before IBP.
	\texttt{ADD\_END} will add or ignore $[...]^{r_{\mathrm{max}}}_{r_{\mathrm{min}}}$
\end{entry}
\subsection{\texttt{class} Gluon\_Integrand}\label{GluonIntegrand}
\begin{entry}{Description}
This usage is almost the same as Laplacian\_Sigma (\ref{LaplacianSigma}). 
This does not involve with interpolation at dipole cross section level. Recommended to use at leaset Chebyshev approximation of $xg(x,Q^2)$.
\end{entry}
\begin{entry}{Constructor}
 Gluon\_Integrand$<$sigma$>$(sigma)
	where sigma is the dipole cross section (\ref{Sigma})
\end{entry}
\begin{entry}{Initialization}
	init(const double* parameters, char mode)\\
	\texttt{parameters} is 1d array containing parameters for $\sigma$. \texttt{mode} is `l' or `s'.  (`l' stands for laplacian, and `s' for sigma (for when laplacian can be computed analytically )). \\
	set\_x($x$) will set $x$ in the components ($\sigma$, $xg$ etc.).
\end{entry}
\begin{entry}{Main function}
	\textbf{operator}()(const double $\rho$, const std::vector$<$double$>$ par) and 
	constant(const double $\rho$, const std::vector$<$double$>$ par)
	are same as Laplacian\_Sigma (\ref{LaplacianSigma}).
\end{entry}

\subsubsection{\texttt{class} Chebyshev\_Laplacian\_Sigma}
Under construction...


\section{dipole-gluon.hh}
\subsection{\texttt{class} Gluon\_GBW}
\begin{entry}{Description}
	Analytic form of GBW dipole gluon density.	
\end{entry}
\begin{entry}{Constructor}
	Gluon\_GBW()
\end{entry}
\begin{entry}{Initialization}
	init(cont double* parameters).
	parameters are the same as those for Sigma (\ref{Sigma}).
\end{entry}
\begin{entry}{Main function}
	\textbf{operator}()($x$,$k_t^2$,$\mu$) 
	compute $\alpha_s\mathcal{F}$. 
	use macro \texttt{ALPHA\_RUN}=1 to use $\alpha_s(\mu^2)$
\end{entry}
\subsection{\texttt{class} Dipole\_Gluon}\label{DipoleGluon}
\begin{entry}{Description}
	compute $\alpha_s\mathcal{F}$.
\end{entry}

\begin{entry}{Constructor}
	Dipole\_Gluon$<$integ$>$(integ).
	for integ  is appropriate integrand (Laplacian\_Sigma~\ref{LaplacianSigma}, Gluon\_Integrand~\ref{GluonIntegrand} etc)
\end{entry}

\begin{entry}{Initialization}
	init(double * parameters)\\
	set\_x(double $x$) 
\end{entry}
\begin{entry}{Main function}
	\textbf{operator}()($x$,$k_t^2$,$\mu$) 
	compute $\alpha_s\mathcal{F}$. 
	use macro \texttt{ALPHA\_RUN}=1 to use $\alpha_s(\mu^2)$
	note Threshold supression $(1-x)^7$ is not added here but already at the point of integrand (e.g. \ref{LaplacianSigma}). 
\end{entry}


\section{Interpolation-gluon.hh}
\subsection{\texttt{class} Approx\_aF}
\begin{entry}{Description}
	Produce grid of $\alpha_s\mathcal{F}(x,k_t^2,0)$.
\end{entry}
\begin{entry}{Constructor}
	Approx\_aF$<$gluon$>$(gluon). where gluon is appropriate gluon to approximated, e.g.~\ref{DipoleGluon}
\end{entry}
\begin{entry}{Initialization}
	init(int n1,int n2, double * parameters).
	n1 number of points for $x$
	n2 number of points for $k_t$.
	parameters to be passed down to $\sigma$.\\
	set\_max(double $k^2_{t,\mathrm{max}}$)	may be replaced in the future. 
	It will generate the grid for predefined limits $-8<\log_{10}(x)<0$, $10^{-6}<k_t^2$ and $k_t^2< k^2_{t,\mathrm{max}}$. 
	In reality the grid is a little wider than specified limits.
\end{entry}
\begin{entry}{Main function}
	\textbf{operator}()(double $x$,double $k^2$, double $\mu^2$)
	computes $\alpha_s\mathcal{F}$
\end{entry}


\end{document}