\documentclass[12pt]{article}
\usepackage[left=1in,right=1in,top=1.5in,bottom=1.5in]{geometry}
\geometry{a4paper}

\usepackage{times}
\usepackage{xcolor}
\usepackage[utf8]{inputenc}
\usepackage{multicol}
\usepackage{amsfonts}
\usepackage{amsmath}
\usepackage{amssymb}
\usepackage{fancyhdr}
\usepackage{tabularx,ragged2e,booktabs,caption}
\usepackage{breqn}
\usepackage{graphicx}
\usepackage{subcaption}
\usepackage{wrapfig}
\usepackage{physics}
\usepackage{hyperref}

\usepackage[
backend=bibtex,
style=ieee
]{biblatex}
\addbibresource{./ref/cambqcd.bib}
\addbibresource{./ref/gbw1998.bib}
\addbibresource{./ref/nikolaev1994.bib}
\addbibresource{./ref/xiao2017.bib}
\addbibresource{./ref/gbk2002.bib}



%\include{./bibliographylist}

\newcommand{\pairdot}[2]{ \mathbf{#1}\cdot\mathbf{#2}  }

\begin{document}
	
\author{Tomoki Goda}
\title{GBW-BK-S Saturation Model}
\date{\today}

%\frontmatter
\maketitle
	
%\tableofcontents
\pagenumbering{arabic}
\numberwithin{equation}{section}
\numberwithin{table}{section}


\section{overall notes, general stuff etc.}

for flavour $f$ and polarization $p$ \cite{cambqcd}\cite{gbw1998},
\begin{equation}
	\sigma_{f,\;p}=\int_{0}^{1} d z \int d^2\mathbf{r}|\Psi_{f,\;p}|^2 \sigma_{DP}.
\end{equation}

The dipole cross section $\sigma_{DP}$ is written \cite{cambqcd}%\cite{gbw1998}
\begin{align}
	\sigma_{DP}&=\int d\mathbf{b}(1-S(\mathbf{r}_1,\mathbf{r}_2)) \\
	&=\sigma_0 (1-\mathcal{S}(\mathbf{r})),
\end{align}
where $\mathbf{b}=\frac{\mathbf{r_1}+\mathbf{r_2}}{2}$, and $S(\mathbf{r}_1,\mathbf{r}_2) =\expval{\tr\left( U^n(\mathbf{r}_1) U^{n\dagger}(\mathbf{r}_2)\right)}/N_c$\footnote{$U ^n (\mathbf{r})=P \exp\left(i g \int^{+\infty}_{-\infty}du n_\mu A^c_\mu (u \mathbf{n}+\mathbf{r}) t^c\right)$. %cf. \cite{cambqcd} p483.
}, and $ \mathbf{r}=(\mathbf{r}_1-\mathbf{r}_2)/2 $ as defined in \cite{cambqcd}.
\newline
Photon wave functions $\Psi_{f,\;p}$ are avaiable in \cite{gbw1998} \cite{nikolaev1994}.

Now, B. W. Xiao et al \cite{xiao2017} defines 
\begin{equation}
x G^{(2)}(x, k_\perp)=\frac{q_\perp^2 N_c}{2 \pi^2 \alpha_s} S_\perp \int \frac{d^2\mathbf{r}_\perp}{(2\pi)^2} e^{-i \mathbf{k}_\perp\cdot \mathbf{r}_\perp}\frac{1}{N_c}\expval{\tr U(0)U^\dagger(\mathbf{r}_\perp)}_x
\end{equation}

Nikolaev et al write\cite{nikolaev1994},
\begin{align}
\sigma(x,r)&=\frac{\pi \alpha_s(r) r^2}{3}\int\frac{d^2 \mathbf{k}}{k^2}\frac{4\left(1-e^{i\mathbf{k}\cdot\mathbf{r}}\right)}{k^2 r^2}\frac{\partial G(x_g,k^2)}{\partial \log(k^2)}\\
&=\frac{4 \pi \alpha_s(r) }{3}\int\frac{d^2 \mathbf{k}}{k^4}\left(1-e^{i\mathbf{k}\cdot\mathbf{r}}\right) \frac{\partial G(x_g,k^2)}{\partial \log(k^2)}
\label{eq:niktot}
\end{align}


where, at leading order \cite{nikolaev1994},
\begin{equation}
\frac{\partial G(x_g,k^2)}{\partial \log(k^2)}=\frac{4}{\pi}\alpha_s(k^2) \left(1-\bra{N}e^{i\mathbf{k}\cdot (\mathbf{r}_1-\mathbf{r}_2)}\ket{N}\right).
\end{equation}



Which is also presented in \cite{gbk2002},
\begin{equation}
\sigma(x,\mathbf{r})=\frac{2\pi}{3}\int\frac{d^2 \mathbf{l}}{l^4}\alpha_s f(x,l^2) (1-e^{i \pairdot{l}{r}})(1-e^{-i \pairdot{l}{r}})
\label{eq:gbktot}
\end{equation}

{\color{red} Something is strange about it, $\frac{\partial G(x_g,k^2)}{\partial \log(k^2)}$ still depends on the direction of $\mathbf{k}$. How can eq. \ref{eq:gbktot} and eq. \ref{eq:niktot} be the same?}% cf. eq(11) of \cite{nikolaev1994}}

Eq. \ref{eq:gbktot} is inverted to yield \cite{gbk2002},
\begin{equation}
\frac{ \alpha_s f(x,l^2)}{l^4}=\frac{3}{4\pi} \int\frac{d^2 \mathbf{r}}{(2\pi)^2} e^{i\pairdot{l}{r}}\left(\sigma(x,\infty)-\sigma(x,r)\right).
\end{equation}

\section{something else}
bessels function of first kind $J_n(z)$ for $n=0$.
\begin{equation}
J_0(z)=\frac{1}{\pi}\int^\pi_0 e^{i z \cos(\theta)}d\theta
\end{equation}
hence,
\begin{equation}
\int d^2\mathbf{r} f(r) e^{i \mathbf{r}\cdot \mathbf{k}}=2\pi \int dr \;r J_0(kr) f(r)
\end{equation}






\newpage
\printbibliography

\end{document}
