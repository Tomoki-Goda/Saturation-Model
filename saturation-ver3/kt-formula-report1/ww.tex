\documentclass{article}

\usepackage[backend=bibtex]{biblatex}
\addbibresource{ww.bib}


\newcommand{\fww}[0]{\mathcal{F}^{\mathrm{WW}}}
\newcommand{\fdp}[0]{\mathcal{F}^{\mathrm{dipole}}}
\newcommand{\sdp}[0]{\sigma_{\mathrm{dipole}}}
\newcommand{\sdpa}[0]{\sigma_{\mathrm{dipole\,Adj.}}}
\begin{document}
\section{Dijet process}
In order to investigate less inclusive processes, we will study the electron-dijet correlation at EIC, following closely the method of Ref.~\cite{vanHameren:2021sqc}. 
Dijet process is known to probe a type of unintegrated gluon density, called Weizs\"acker-Williams gluon density\cite{}.
This UGD, $\fww(x,k^2)$, has an interpretation as a number density of gluons inside a proton. Under the Gaussian approximation and assuming $\theta$-like profile of the proton, one can write  
\begin{equation}
\fww(x,k^2)= \frac{C_F}{2\pi^3\alpha_s}\int^\infty_0\frac{dr}{r}J_0(r k) \sdpa(x,r),
\end{equation} 	
with the adjoint dipole cross section
\begin{equation}
\sdpa(x,r)=\sigma_0\left( 1-\left(1-\frac{\sdp(x,r)}{\sigma_0}\right)^{C_A/C_F}\right).
\end{equation}
Resummation of the Sudakov logarithms is achieved by the formula~\cite{}
\begin{equation}
	\fww(x,k^2,\mu^2)= \frac{C_F}{2\pi^3\alpha_s}\int^\infty_0\frac{dr}{r}J_0(r k) e^{-S(r,\mu^2)} \sdpa(x,r),
\end{equation}
where, we use the Sudakov form factor~\cite{},
\begin{equation}
	S(r,\mu^2)=\frac{\alpha_s N_c}{4\pi}\ln^2\left(\frac{\mu^2r^2}{4e^{-2\gamma_E}}\right),
\end{equation}
in which $\gamma_E$ is the Euler-Mascheroni constant, and we use $\alpha_s=0.2$. 



	
	
	
\appendix	
\section{Computation of gluon densities}
In order to obtain the WW gluon density, one needs to compute 
\begin{equation}
	\fww(x,k)= \frac{C_F}{2\pi^3\alpha_s}\int^\infty_0\frac{dr}{r}J_0(r k) \sdpa(x,r),
\end{equation} 	
where we use 
\begin{equation}
	\sdpa(x,r)=\sigma_0\left( 1-\left(1-\frac{\sdp(x,r)}{\sigma_0}\right)^{C_A/C_F}\right).
\end{equation}

The convergence of this integral may be slow due to high oscillation for large $k$.
For such integration, one can employ a method summarized in Ref.~\cite{LYNESS1985109}. 
Essentially, the integration is carried out between, anti-nodes of $J_0(rk)$. Then the total integral is written in the form of infinite sum
\begin{equation}
	I=\sum^\infty_{i=1} \int^{a_{i+1}}_{a_i} \frac{dr}{r}J_0(r k) \sdpa(x,r),
\end{equation}
where we use $a_{i}=\frac{\pi}{k}(i+1/4)$.
This sum can be accelerated, with series transformation described in Ref.~\cite{doi:10.1080/00207167308803075,Weniger:1989rea,HOMEIER19951}.
The remarkable power of the Levin's transformation is shown in Tab.~\ref{tab:levin}.
This trick is also useful in the computation of dipole gluon density. 

\begin{table}
	\begin{center}
	\begin{tabular}{|c||c|c|}
		\hline
		$n$&Levin&Sum\\\hline
		9&	-5.751e-9&	-0.0004998\\\hline
		12&	-1.5891e-9&	0.00025567\\\hline
		15&	-3.803e-11&	-0.00015069\\\hline
		18&	6.639e-13&	0.00009743\\\hline
		21&	2.6728e-13&	-0.00006722\\\hline
		24&	2.6744e-13&	0.00004866\\\hline
		27&	2.6744e-13&	-0.00003655\\\hline
	\end{tabular}
	\end{center}
	\caption{
		Computing $\int \frac{dr}{r}J_0(r k)(1-e^{-r^2})$, with $k=10$ adding up to $n$ th term.
		Levin-accelerated case reaches the true value of $\sim$2.674e-13 far more quickly, and shows superior stability.
	}
\label{tab:levin}
\end{table}
\printbibliography
\end{document}

